\cleardoublepage
\chapternonum{摘要}
智能合约作为计算机程序可能包含编程错误或漏洞,这种安全隐患会给参与者带来经济损失和信任危机。而且,智能合约被部署到区块链之后便不可再修改,这就导致包含错误或漏洞的合约将永远被保留在区块链上,占据存储资源且影响区块链的运行效率。因此在智能合约被部署到区块链上之前对其进行安全分析和检测是十分必要的。

迄今已有许多针对智能合约漏洞的检测方法被提出来,如模糊测试法、符号执行法等。然而前者往往需要大量的计算资源,后者需要遍历合约所有可能的执行路径,这些方法不仅耗时而且效率不佳。相比于传统方法,深度学习法为智能合约漏洞检测提供了一种高效的解决方案,其核心思想是提取智能合约的漏洞特征来训练深度学习模型,以实现检测智能合约漏洞的目的。然而现有工作大都只关注单一类型的特征,即仅使用专家特征或语义特征,少有工作同时考虑到这两种特征,另一方面,近年来兴起的预训练模型在漏洞检测任务中表现出色,而在智能合约领域鲜有应用。鉴于此,本文提出了一种全新的基于代码领域的预训练模型、并且融合专家特征和语义特征进行智能合约漏洞检测的方法。总体来说,本文的研究内容和结果如下所述:

    1)基于专家特征和语义特征融合的智能合约表征。

    本文探究了一些契合智能合约源代码的静态代码指标,作为表达智能合约结构信息的专家特征,同时从 AST 中提取三种语义图作为智能合约的语义特征,然后融合两种特征开展漏洞检测研究。消融实验结果表明,当移除专家特征时,三种指标 Recall、Precision、F1-score 分别下降的百分比为 4.12\%、6.06\%、5.86\%, 这表明静态代码指标可以在一定程度上能提升漏洞检测效果,但效果并不显著;而当移除语义特征时,三种指标分别下降的百分比为 34.06\%、29.35\%、31.08\%,这表明语义信息可以显著提高模型对智能合约的学习能力,提升智能合约漏洞检测效果。

    2)基于预训练模型的智能合约漏洞检测方法研究。
    
    本文使用了深度学习法对智能合约的静态代码指标和语义图进行嵌入,以生成其在高维空间的向量表示。另一方面,本文将GraphCodeBERT作为基础模型结合上一步得到的高维智能合约表征,并搜集了大规模的数据集用于训练智能合约漏洞检测模型,然后设计实验对本文提出的方法进行验证,实验结果表明,本文提出的方法能有效检测出智能合约中常见的算术漏洞、权限控制漏洞、重入漏洞和异常调用漏洞,并且在\num{54739}份智能合约上的评估指标Recall、Precision、F1-score分别达到了0.90、0.97、0.93,优于另外两种基线方法Slither和DR-GCN。
    
    3)智能合约运行时崩溃预测方法研究。
    
    针对目前智能合约运行时崩溃导致交易失败的相关研究较少的情况,本文在上一步工作的基础上,丰富智能合约数据集中交易结果相关的信息,并利用漏洞检测模型对智能合约运行结果进行预测,实验结果表明本文的方法可以在一定程度上预测出智能合约在运行时是否会崩溃,评估指标Precision、Recall、F1-score分别为0.83、0.76、0.79。最后探究了智能合约的各种漏洞与运行时崩溃的关系,结果表明在本文研究的四种漏洞中,算术漏洞最有可能导致合约在运行时崩溃。



    {\noindent \textbf{关键词:} 智能合约、漏洞检测、深度学习}
\cleardoublepage
\chapternonum{Abstract}
Smart contracts, as computer programs, may contain programming errors or vulnerabilities, a security risk that can cause financial losses and a crisis of trust for participants. Moreover, smart contracts cannot be modified after they are deployed on the blockchain, which results in contracts containing errors or vulnerabilities being retained on the blockchain forever, occupying storage resources and affecting the operational efficiency of the blockchain. Therefore, it is necessary to analyze and test the security of smart contracts before they are deployed on the blockchain.

So far, many methods have been proposed for detecting smart contract vulnerabilities, such as fuzzy testing method, symbolic execution method, etc. However, the former method often requires a lot of computation. However, the former often requires a large amount of computational resources, and the latter needs to traverse all possible execution paths of the contract, which are not only time-consuming but also inefficient. Compared with traditional methods, deep learning method provides an efficient solution for smart contract vulnerability detection, and its core idea is to extract the vulnerability features of smart contracts to train deep learning models for the purpose of detecting smart contract vulnerabilities. However, most of the existing works focus on a single type of features, i.e., using only expert features or semantic features, and few works consider both of them. On the other hand, the pre-trained models emerging in recent years perform well in the task of vulnerability detection, while they have rarely been applied in the field of smart contracts. In view of this, this paper proposes a novel approach to smart contract vulnerability detection based on pre-trained models in the code domain and fusing expert features and semantic features. Overall, the research content and results of this paper are described as follows:

    1) Smart contract characterization based on the fusion of expert features and semantic features.

    In this paper, we explore some static code metrics that fit the source code of smart contracts as the expert features that express the structural information of smart contracts, and at the same time, we extract three kinds of semantic graphs from ASTs as the semantic features of smart contracts, and then fuse the two kinds of features to carry out the vulnerability detection research. The results of the ablation experiment show that when the expert features are removed, the percentages of the three indicators Recall, Precision, and F1-score decrease by 4.12\%, 6.06\%, and 5.86\%, respectively, which indicates that the static code indicators can improve the vulnerability detection effect to a certain extent, but the effect is not significant; while when the semantic features are removed, the percentages of the three indicators decrease by 34.06\%, 6.06\%, and 5.86\%, respectively. When the semantic features are removed, the percentages of the three indicators are 34.06\%, 29.35\%, 31.08\%, which indicates that the semantic information can significantly improve the model's learning ability of the smart contract, and improve the effect of smart contract vulnerability detection.

    2) Research on smart contract vulnerability detection method based on pre-trained model.
    
    In this paper, a deep learning method is used to embed static code metrics and semantic graphs of smart contracts to generate their vector representations in high-dimensional space. On the other hand, this paper combines GraphCodeBERT as the base model with the high-dimensional smart contract representation obtained in the previous step, and collects a large-scale dataset for training the smart contract vulnerability detection model, and then designs experiments to validate the method proposed in this paper, and the experimental results show that the method proposed in this paper can effectively detect the common arithmetic vulnerabilities, privilege control vulnerabilities, re-entry vulnerabilities, and exception call vulnerabilities in smart contracts. entry vulnerabilities and exception call vulnerabilities, and the evaluation metrics Recall, Precision, and F1-score on \num{54739} smart contracts reach 0.90, 0.97, and 0.93, respectively, which are better than the other two baseline methods Slither and DR-GCN.
    
    3) Research on smart contract runtime crash prediction methods.
    
    Aiming at the current situation that there are fewer studies related to transaction failure caused by smart contract crash at runtime, this paper enriches the information related to transaction result in the smart contract dataset on the basis of the previous work and predicts the smart contract runtime result by using the vulnerability detection model, and the experimental results show that the method of this paper can predict to a certain extent whether the smart contract will crash at runtime, and the evaluation indexes Precision, Recall, and F1-score are 0.83, 0.76, and 0.79, respectively.Finally, the relationship between various vulnerabilities of smart contracts and runtime crashes is explored, and the results show that among the four kinds of vulnerabilities studied in this paper, the arithmetic vulnerability is the most likely to lead to the crash of the contract at runtime.



    {\noindent \textbf{Key words: } Smart Contracts, Vulnerability Detection、Deep Learning}
