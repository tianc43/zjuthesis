\cleardoublepage
\chapternonum{摘要}
基于区块链的智能合约遵循“去中心化、不可修改”的特性,避免了信任和隐私安全等问题。但智能合约可能包含编程错误或漏洞,这种安全隐患可能会给参与者带来严重的经济损失和信任危机。另一方面,智能合约被部署到区块链之后便不可再修改,除非自我销毁,这就导致包含错误或漏洞的合约将永远被保留在区块链上,占据存储资源且影响区块链的运行效率。因此在智能合约被部署到区块链上之前对其进行安全分析和检测是十分必要的。

迄今已有许多漏洞检测工具或方法被提出来。但是,传统的漏洞检测方法往往需要大量的计算资源(如模糊测试法),有时甚至需要遍历所有可能的执行路径或达到一定的分析深度(如符号执行法),这不仅耗时而且效率不佳。相对而言,近年来兴起的基于预训练模型的方法通过自动化提取合约特征,并结合精心设计的预训练任务进行训练,最终得到的漏洞检测模型能显著提升检测效率和准确度。然而现有工作大都只关注单一特征,比如静态分析法只利用了智能合约的静态代码特征,而深度学习法只关注源代码本身的语义信息,还没有工作能同时考虑到这两种特征,并融合两种特征进行漏洞检测模型的训练。鉴于此,本文提出了一种全新的基于代码领域的预训练模型、并且融合专家特征和语义特征进行智能合约漏洞检测的方法。总体来说,本文的研究内容和结果如下所述:

    1)基于特征融合的智能合约漏洞检测方法研究。
    
    本文使用了深度学习法对智能合约的静态代码指标和语义图进行嵌入,以生成其在高维空间的向量表示。另一方面,本文将GraphCodeBERT作为基础模型结合上一步得到的高维智能合约表征,并搜集了大规模的数据集用于训练智能合约漏洞检测模型,然后设计实验对本文提出的方法进行验证,实验结果表明,本文提出的方法能有效检测出智能合约中常见的算术漏洞、权限控制漏洞、重入漏洞和异常调用漏洞,并且在\num{54739}份智能合约上的评估指标Recall、Precision、F1-score分别达到了0.90、0.97、0.93,优于另外两种基线方法Slither和DR-GCN。同时消融实验结果表明,当移除专家特征时,三种指标Recall、Precision、F1-score分别下降的百分比为4.12\%、6.06\%、3.86\%;而当移除语义特征时,三种指标分别下降的百分比为34.06\%、29.35\%、31.08\%,这表明静态代码特征和语义特征都能提升漏洞检测模型的性能,而融合两者训练得到的模型能获得最大的性能提升。
    
    2)智能合约运行时崩溃预测方法研究。
    
    针对目前智能合约运行时崩溃导致交易失败的相关研究较少的情况,本文在上一步工作的基础上,丰富智能合约数据集中交易结果相关的信息,并利用漏洞检测模型对智能合约运行结果进行预测,实验结果表明本文的方法可以在一定程度上预测出智能合约在运行时是否会崩溃,评估指标Precision、Recall、F1-score分别为0.83、0.76、0.85。最后探究了智能合约的各种漏洞与运行时崩溃的关系,结果表明在本文研究的四种漏洞中,算术漏洞最有可能导致合约在运行时崩溃。



    {\noindent \textbf{关键词:} 智能合约、漏洞检测、深度学习}
\cleardoublepage
\chapternonum{Abstract}
Blockchain-based smart contracts follow the characteristics of "decentralization and immutability", which avoids issues such as trust and privacy security. However, smart contracts may contain programming errors or vulnerabilities, and such security risks may cause serious economic losses and trust crises for participants. On the other hand, smart contracts cannot be modified after they are deployed to the blockchain unless they are self-destructed, which results in contracts containing errors or vulnerabilities being retained on the blockchain forever, occupying storage resources and affecting the operational efficiency of the blockchain. Therefore, it is necessary to analyze and test the security of smart contracts before they are deployed on the blockchain.

Many vulnerability detection tools or methods have been proposed so far. However, traditional vulnerability detection methods often require a large amount of computational resources (e.g., fuzzy testing method), and sometimes even need to traverse all possible execution paths or reach a certain depth of analysis (e.g., symbolic execution method), which is not only time-consuming but also inefficient. However, most of the existing work only focuses on a single feature, for example, static analysis method only utilizes the static code features of smart contracts, while deep learning method only focuses on the semantic information of the source code itself, and there is no work that can take both features into account and integrate both features for vulnerability detection model training. In view of this, this paper proposes a novel approach for smart contract vulnerability detection based on a pre-trained model of the code domain and fusing expert features and semantic features. Overall, the research content and results of this paper are described as follows:

    1) Research on smart contract vulnerability detection method based on feature fusion.
    
    In this paper, a deep learning method is used to embed static code metrics and semantic graphs of smart contracts to generate their vector representations in high-dimensional space. On the other hand, this paper combines GraphCodeBERT as the base model with the high-dimensional smart contract representation obtained in the previous step, and collects a large-scale dataset for training the smart contract vulnerability detection model, and then designs experiments to validate the method proposed in this paper, and the experimental results show that the method proposed in this paper can effectively detect the common arithmetic vulnerabilities, privilege control vulnerabilities, re-entry vulnerabilities, and exception call vulnerabilities in smart contracts. entry vulnerabilities and exception call vulnerabilities, and the evaluation metrics Recall, Precision, and F1-score on \num{54739} smart contracts reach 0.90, 0.97, and 0.93, respectively, which are better than the other two baseline methods Slither and DR-GCN.Meanwhile, the results of the ablation experiments show that, when removing the expert features, the three metrics Recall, Precision, and F1-score decrease by 4.12\%, 6.06\%, and 3.86\%, respectively; while when the semantic features are removed, the three metrics decrease by 34.06\%, 29.35\%, and 31.08\%, respectively, which indicates that both the static code features and the semantic features can improve the performance of the vulnerability detection model. This indicates that both static code features and semantic features can improve the performance of the vulnerability detection model, and the model obtained by fusing the two training can get the maximum performance improvement.
    
    2) Research on smart contract runtime crash prediction method.
    
    Aiming at the current situation that there are fewer studies on transaction failure caused by smart contract runtime crash, this paper enriches the information related to transaction result in smart contract dataset on the basis of the previous work, and predicts the runtime result of smart contract by using vulnerability detection model, and the experimental results show that this paper's method can predict to a certain extent whether the smart contract will crash during runtime, and the evaluation indexes are as follows Precision, Recall, and F1-score are 0.83, 0.76, and 0.85, respectively.Finally, the relationship between various vulnerabilities of smart contracts and runtime crashes is explored, and the results show that among the four kinds of vulnerabilities studied in this paper, the arithmetic vulnerability is the most likely to lead to the crash of the contract at runtime.



    {\noindent \textbf{Key words: } Smart Contracts, Vulnerability Detection、Deep Learning}
