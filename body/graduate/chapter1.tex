\chapter{绪论}
\section{本章内容}
本章首先简单介绍了区块链技术和智能合约的发展历程,以及智能合约漏洞可能导致的风险,以现实世界的实例说明了智能合约中的漏洞造成的经济损失和信任危机。针对智能合约中潜在的漏洞,现有的应对之策包括:第三方专业机构对其进行审计和担保、借助漏洞检测技术对智能合约进行分析,然而第三方机构高昂的审计费用,以及尚没有得到广泛应用的漏洞检测技术,都在促使研究人员探索出一种简单高效的漏洞检测方法或工具。


\section{研究背景和意义}

区块链(Blockchain)最初由中本聪(Satoshi Nakamoto)在2008年提出,被应用在比特币的创造和交易中。区块链是一种分布式数据库,其核心概念是去中心化和安全性,它通过将数据存储在多个节点上,并使用密码学技术确保数据的安全性和完整性。区块链的每个区块都包含了前一个区块的信息以及时间戳,形成了一个不可篡改的链式结构,使得所有的交易记录都可以被公开验证和追溯。这种技术的应用不仅限于加密货币领域,还可以应用于金融服务、供应链管理、物联网、医疗保健等领域。

随着区块链技术的发展,智能合约(Smart Contract)应运而生。智能合约是一种以代码形式编写的协议,类似于现实世界中的“合同”。而智能合约可以在没有第三方参与的情况下执行交易,实现自动化的合约执行和规则执行。智能合约的概念最早由尼克·萨博(Nick Szabo)提出,随后由以太坊(Ethereum)创始人维塔利克·布特林(Vitalik Buterin)在以太坊平台上得到了广泛的应用。智能合约的出现使得去中心化的应用成为可能,它们可以自动执行协议,管理数字资产,实现不同方之间的信任,并确保交易的安全性和可靠性。

智能合约的出现极大地推动了区块链技术的应用,其自动执行以及不可篡改的特性也导致合约的安全性面临着一系列潜在缺陷,这些缺陷可能带来严重的信任风险和财务损失。事实上,自智能合约问世以来,其代码缺陷导致的金融安全事件层出不穷。2016年,基于以太坊区块链的分散自治组织(The DAO)发生了一起严重的智能合约漏洞事件。攻击者利用智能合约的重入漏洞成功发起了一次针对The DAO的攻击,窃取了价值数百万美元的加密货币。此次事件导致以太坊价格暴跌,并引发了一场有关区块链安全性的广泛讨论。2017年,基于以太坊的多重签名钱包服务提供商Parity Technologies发生了一起严重的智能合约漏洞事件。由于智能合约中的一个错误操作,攻击者成功冻结了价值数百万美元的加密货币资产,影响了众多加密货币投资者和交易所的资金安全。2020年,去中心化交易协议UniSwap中发现了一处智能合约漏洞,导致攻击者利用闪电贷功能成功实施了欺诈。攻击者利用该漏洞进行了大规模交易,导致大量加密货币资产流失。2023年3月,金融借贷平台Euler Finance被黑客攻击。相关合约中的一个关键函数缺少清算检查逻辑,因此攻击者无需任何抵押即可转移清算资金,该漏洞被黑客利用后直接导致价值约1.97亿美元的加密货币被盗。

以上事件表明智能合约漏洞严重损害了区块链社区对于智能合约安全性的信任,可能给参与者带来严重的经济损失和信任危机,进一步凸显了智能合约安全性的重要性。为了提高智能合约的安全性,已经诞生了一些专业的第三方机构,如Cyfrin、OpenZeppelin、慢雾科技等。他们会对智能合约代码进行全面的审计和分析,以识别低效代码以及潜在的漏洞,最终出具一份专业的审计报告。开发者可以凭此审计报告声明智能合约的安全性,\underline{进而获得区块链社区以及用户的信任,为自己的应用进行担保}。然而不幸的是,专业机构通常需要收取高昂的审计费用,这往往让大多数开发者望而却步。\underline{因此,探索另外的方式是十分必要的}。


事实上,从智能合约诞生以来,借助软件工程领域的技术对智能合约进行漏洞检测(Vulnerability Detection)的研究从未停止。然而迄今还没有出现一种漏洞检测方法或工具被众多区块链社区采纳,\underline{因此本文对漏洞检测技术的研究是十分必要的}。


\section{国内外研究现状}
区块链以其安全性备受关注,然而智能合约中的漏洞所导致的安全问题却在一定程度上制约了区块链应用的流行,这种矛盾吸引了开发者和研究人员对智能合约漏洞检测技术的探索。目前已经诞生了多种漏洞检测方法,主要包括三类:静态分析法、动态分析法和机器学习法。

\subsection{漏洞检测方法}

静态分析法是指在不运行智能合约的前提下进行漏洞检测,主要包括以下方法:形式化验证法、符号执行法、污点分析法;动态分析法则是指在智能合约运行过程中进行漏洞检测,主要指模糊测试法;机器学习法主要包括利用传统的机器学习模型,以及利用人工神经网络模型对智能合约进行漏洞检测的方法。
\begin{itemize}
    \item \textit{\textbf{形式化验证法}}。形式化验证法利用数学语言和逻辑表达式,将智能合约中的逻辑关系转换为形式规范,通过数学推理和证明,发现可能存在的逻辑错误、漏洞和不安全的设计,从而提前识别潜在的风险。然而,形式化验证法也面临一些挑战,比如对于复杂合约的处理可能较为繁琐,且需要深厚的数学和形式化知识。
    \item \textit{\textbf{符号执行法}}。符号执行法将合约中的变量和输入符号化,用符号代替实际数值,合约的执行路径就不再依赖于具体的输入,研究人员就可以基于符号值进行推演,这使得符号执行法能够遍历多个可能的执行路径。同时,通过对符号值的范围进行分析,可以发现一些算术异常或者条件不当的分支等问题。然而,符号执行法也面临路径爆炸和对合约复杂性的适应性等问题。
    \item \textit{\textbf{污点分析法}}。污点分析法将合约中的敏感数据视为“污点”,如转账金额、用户身份信息、用户输入,并追踪这些敏感数据在合约内的函数调用或变量操作,从而揭示潜在的漏洞和风险。然而,在面临复杂的数据流和动态调用的合约逻辑时,该方法识别出潜在漏洞的能力也会下降。
    \item \textit{\textbf{模糊测试法}}。模糊测试法会生成大量具有随机性的输入数据,模拟各种可能的用户输入,包括超出正常范围的数值、或者其他可能导致合约异常行为的情况,从而发现智能合约可能存在的安全漏洞。然而,这种方法也有一些局限性,比如需要大量的测试用例、难以涵盖所有可能的输入情况以及可能产生大量误报的问题。
    \item \textit{\textbf{机器学习法}}。机器学习法可以从包含已知漏洞的数据样本中学习到智能合约的关键特征,从而在新的合约中识别潜在的安全问题。特别是深度学习模型,如循环神经网络(RNN)或长短时记忆网络(LSTM),能够更好地应对合约中复杂的逻辑和数据流。相比于上述其他方法,机器学习法为智能合约漏洞检测提供了一种创新的、高效的解决方案。
\end{itemize}

\textcolor{red}{这里可能需要扩充}

\par \autoref{tab:defect_detection_methods}中列出了每一种智能合约漏洞检测方法的名称、相关的论文数量及代表模型。

\begin{table}[htbp]
    \caption{\label{tab:defect_detection_methods}现有的智能合约漏洞检测方法}
    \small
    \begin{threeparttable}
        {
            \renewcommand{\arraystretch}{1.5}
        \begin{tabularx}{\linewidth}{cX<{\centering}X<{\centering}X<{\centering}}
            \toprule
            类别                     & 方法名称   & 相关论文数量 & 代表方法 \\ \midrule
            \multirow{3}{*}{静态分析法} & 形式化验证法 & 111    & aaa  \\
                                   & 符号执行法  & 222    & bbb  \\
                                   & 污点分析法  & 333    & ccc  \\ \midrule
            动态分析法                  & 模糊测试法  & 444    & ddd  \\ \midrule
            \multirow{2}{*}{机器学习法} & 传统机器学习法  & 555    & eeee \\
                                   & 深度学习法  & 666    & fff  \\ \bottomrule
        \end{tabularx}
        }
        \begin{tablenotes}
            \footnotesize
            \item[*] 每种方法相关的论文数量统计范围为\underline{2015年至2023年}
        \end{tablenotes}
    \end{threeparttable}
\end{table}

\subsection{相关工作}
\begin{enumerate}[label=\textbf{\textit{\Alph*.}}, align=left]
    \item 形式化验证法
    
    2018年,Grishchenko等人\cite{grishchenko2018semantic}提出了以太坊虚拟机(Ethereum Virtual Machine,EVM)字节码的首个小步骤语义(Small-Step Semantics),然后在F-star框架中进行形式化得到可执行代码,并成功在官方测试套件中验证通过。同年,Hildenbrandt等人先后提出了他们自己的智能合约检测工具KEVM框架、Isabelle/HOL [9]和Zeus。KEVM框架是基于字节码堆栈的EVM语言的可执行形式规范,由K框架构建。Isabelle/HOL是一种通过字节码级的合理程序逻辑扩展现有EVM形式化的检测工具,然后将字节码序列构建为线性代码块,并创建一个程序逻辑来推断这些代码。ZEUS是一种检测工具,利用抽象解释、符号模型检查和约束语句快速测试智能合约的安全性。它在以太坊区块链平台上构建了一个原型,评估了超过22.4k的智能合约。评估结果显示,约94.6\%的合同容易受到攻击。2019年,Lianan等人推出了一个智能合约的形式验证平台Vaas。它为智能合约提供了安全检测和验证,准确定位风险代码并提出修改建议,自动检测智能合约的10个主要和32个次要的传统安全漏洞和功能逻辑缺陷。它还支持对安全检测的蚂蚁链、腾讯区块链、ETH、EOS、ONT、TRON、fisco-bcos、fabric等区块链平台的支持。
    \item 符号执行法
    
    符号执行方法的核心思想是对智能合约代码中的不确定值变量进行符号化处理,然后逐步分析、解释和执行程序中的指令,并在过程中更新执行状态并搜索路径约束。同时,分支节点执行分支执行,完成程序中所有可执行路径的检测,以发现漏洞。2016年,Luu等人提出了一个名为Oyente的智能合约检测工具,这是一种符号执行工具,用于查找潜在的智能合约安全漏洞。2017年,开发了一种名为Mythril的智能合约漏洞检测工具。Mythril检测工具是以太坊的官方智能合约漏洞检测工具。它可以使用符号执行来探索所有可能的不安全路径,以检测14种智能合约漏洞,如时间戳依赖性、任意地址写入和整数溢出。2018年,Nikolic等人提出了一种名为Maian的智能合约检测工具,该工具是一种用于规范和推理跟踪属性的工具。它使用程序间符号分析和具体验证器显示真实的调用关系。Maian主要关注三类智能合约漏洞,包括不定资产冻结、资产泄露和合同销毁。同年,Tsankov等人先后提出了他们自己的智能合约漏洞检测工具securify、teether和sereum。Securify是一种可扩展且完全自动化的以太坊智能合约漏洞检测工具,可以证明合同行为对于给定内容是否安全。TeEther是通过开发易受攻击的合同的通用定义构建的。Sereum基于EVM字节码指令级运行时监视来检测重入漏洞。
    Oyente是由Luu等人提出的第一批使用符号执行检测基于合同控制流图的智能合约缺陷的工具之一。该工具支持检测重入漏洞、异常处理漏洞、交易顺序依赖漏洞等。可以看出,该工具可以检测较少类型的缺陷。随后的许多工作对Oyente进行了扩展,例如Maian,进一步扩展了检测的缺陷类型。Mythril也是一种基于符号执行的检测工具。它使用合同字节码中的不同事务来探索合同的状态空间。当合同处于不希望的状态时,它确定合同存在缺陷。然而,该方法存在路径可达性问题,这是符号执行固有的。这意味着该方法假定所有探索的路径都是可达的。基于符号执行的检测方法存在路径可达性问题。由于这个问题,该方法会忽视一些漏洞,降低了检测的准确性。而且由于需要访问所有可能的路径,因此它耗时,因此在该方法中很难在检测效率和准确性之间取得平衡。
    
    \item 污点分析法
    
    污点分析方法的核心思想是在程序中识别并标记污点信息,通过使用特定规则分析污点源,追踪污点传播路径,并判断其是否受到无害处理,最终实现漏洞检测的目的。2018年,Torres等人提出了结合污点分析和符号执行的智能合约安全分析框架Osiris。该框架将输入合同源代码编译成EVM字节码,利用符号分析原理构建控制流图以实现每个指令结果的传输,分析和跟踪内存和存储中污点的传播路径,检测整数错误,并实现智能合约漏洞的检测。2019年,Gao等人提出了一个名为Easyflow的智能合约安全分析框架,主要针对智能合约溢出漏洞。Easyflow工具基于EVM水平,利用污点分析组件跟踪和分析污点,监视交易过程,实现检测智能合约漏洞的目的。
    
    \item 模糊测试法
    
    模糊测试方法的核心是将生成的数据输入程序,检测程序是否异常运行,以探测潜在漏洞。这一技术在智能合约领域的研究也逐渐成熟。2018年,蒋等人提出了一个名为ContractFuzzer的智能合约检测工具。ContractFuzzer通过EVM日志工具记录智能合约运行时的行为,并分析这些日志以报告安全漏洞。同年,刘等人提出了一个名为ReGuard的智能合约检测工具,主要针对智能合约的潜在重入漏洞。ReGuard通过迭代生成随机且多样化的测试数据,可在合同运行时进行监控,并实现对重入漏洞的动态识别。2019年,何等人提出了一个名为ILF的智能合约漏洞检测工具。它是一种端到端的智能合约漏洞检测系统,并实例化了处理模糊智能合约问题的方法。2020年,Torres等人提出了一种名为ConFuzzius的检测工具,该工具基于混合模糊测试结合符号执行和模糊测试检测以太坊智能合约的安全漏洞。ConFuzzius通过模糊测试检测智能合约的浅层代码,并通过符号执行和约束求解生成满足复杂条件的输入,从而探测智能合约的深层代码。2021年,崔等人提出了另一个检测工具SMARTIAN,它是一个实用的开源模糊测试工具。它使用智能合约字节码的静态分析来预测交易顺序,并决定每个交易是否应满足某些约束。在模糊测试阶段,该工具利用先前的信息构建初始种子语料库,并执行动态数据流分析,根据数据流有效地引导模糊测试。
    基于模糊测试的方法。姜等人提出了ContractFuzzer,这是一种用于智能合约缺陷动态检测的工具。它采用模糊测试方法,生成模糊输入并定义测试场景以检测缺陷。该工具配置EVM并在智能合约运行时记录其行为日志。然后分析这些日志以发现缺陷。刘等人提出了ReGuard,用于检测重入缺陷的模糊测试工具,它将Solidity语言中的智能合约转换为C++,并基于有限状态机模型生成交易以检测缺陷。然而,Solidity语言和C++语言是两种不同的语言。转换过程不可避免地导致语法和语义信息的丢失。还有Nguyen等人提出的sfuzz,这是一个改进的用于模糊输入生成的工具,但它没有充分利用运行时数据并具有一定的局限性。基于模糊测试的检测方法在生成测试用例方面需要进一步优化。否则,很难覆盖程序运行的所有路径,使得一些复杂的漏洞路径难以达到,从而导致检测准确性较低。

    \item 机器学习法
    
    深度学习方法的核心思想是通过提取各种智能合约的漏洞特征来训练深度学习模型,以实现检测智能合约漏洞的目的。2018年,Tann等人提出了一种名为SaferSC的智能合约漏洞检测工具,它是一种利用机器学习方法(长短时记忆网络LSTM)快速检测智能合约漏洞的工具。2020年,Zhuang等人提出了一种名为DR-GCN和TMP的智能合约检测工具,它是一种用于检测以太坊和Viterbi链上智能合约的三种漏洞的工具:重入、时间戳依赖性和无限循环,通过图神经网络实现。DR-GCN检测工具通过构建智能合约图表示智能合约函数的语法和语义结构,并通过一种无量纲图卷积神经网络检测智能合约漏洞。TMP通过构建智能合约图表示智能合约函数的语法和语义结构,并通过新的时间信息传播网络学习检测智能合约漏洞。同年,Wang等人提出了一种名为ContractWard的智能合约漏洞检测工具。该工具首先提取智能合约简化操作代码的二进制语法特征,然后使用多种机器学习算法和采样算法建立漏洞检测模型。2021年,Liu等人提出了一种名为AME的智能合约漏洞检测工具,它结合了深度学习和专家模式。
    基于机器学习的方法。王等人采用机器学习方法来检测智能合约缺陷。他们从智能合约的操作码中提取二元特征,并使用各种机器学习算法和采样算法来检测智能合约缺陷。该工作支持检测包括重入漏洞、整数溢出漏洞、时间戳依赖漏洞等在内的六种类型的缺陷。这种方法显著节省了检测时间。但是,本文中使用的智能合约数据集仅由一个Oyente工具检测,该工具也存在误报和漏报。因此,他们工作中数据集的置信水平较低。深度学习是机器学习的一种类型,一些研究表明深度学习对于代码缺陷检测具有益处,并且已经有一些深度学习方法应用于智能合约缺陷检测。钱等人提出了ReChecker,这是智能合约中检测重入漏洞的第一个基于深度学习的方法。ReChecker需要将智能合约源代码转换为合同块,并捕捉合同中的基本语义和控制流依赖信息。该工具使用双向长短时记忆模型(BiLSTM)和注意力机制自动检测智能合约中的重入漏洞。这种方法只实现了一种缺陷的检测。然而,已经有数十种已知的缺陷类型,需要扩大缺陷的范围。还有一些基于深度学习的技术用于智能合约缺陷检测,例如DR-GCN,它将智能合约转换为图结构,然后将合同图放入图神经网络以检测缺陷。DR-GCN提取合同的关键信息,从而显著提高了检测效率,但该模型仅支持三种缺陷检测,并检测的缺陷较少。基于机器学习的检测方法显著节省了检测时间,并提高了智能合约缺陷检测的性能。然而,在缺陷覆盖方面仍存在不足,有些工作只测试了三种或六种缺陷。扩大缺陷覆盖范围仍然是智能合约缺陷检测工作中需要解决的问题。此外,在数据集处理期间还需要进行优化,以提高数据集的质量和可信度。
\end{enumerate}
\textcolor{red}{这里需要扩充一下,说明虽然静态和动态分析法已经有非常多的研究工作,也有广为人知的工具。但是在某些方面还是存在劣势,而机器学习法正好能够弥补这些不足,所以探究更高效的机器学习法是十分必要且有意义的}

\section{本文的研究内容}

\section{本文的组织结构}