\chapter{智能合约运行时崩溃预测的尝试性探索}
\section{本章内容}
当前,关于智能合约在运行时发生异常而崩溃的研究寥寥无几,因此本文在前文建立的数据集和漏洞检测模型的基础上,对智能合约的运行时崩溃预测进行了研究。
\section{方法概述}
如果能在发送交易前,根据当前状态信息对交易结果进行预测,提前发现可能失败的交易,便能提高以太坊的运行效率,同时节省交易手续费。
合约在运行时的异常
会导致存储浪费和影响执行效率
智能合约被部署到以太坊网络上后,其源代码便不可再更改,因此只要该合约被调用,那么它的执行逻辑就不会改变。
\section{实验数据集}
\textcolor{red}{这里可能要补充一些运行时的信息作为特征,不然可能会被喷:为什么同样的特征既可以用来检测漏洞也可以用来检测运行时异常?}
在数据集方面,我们沿用了从前文搜集的合约中提取出的专家特征和语义特征,但在标签上有一些不同。

智能合约在运行时可能发生的异常如xxx节所述,包括Gas不足、reverted、无效操作码、无效跳转和堆栈溢出。
下文将在运行时刻发生异常而崩溃的合约称为崩溃合约。

以太坊中记录了智能合约的所有交易的执行结果,因此智能合约在运行时是否因发生异常而崩溃等信息都可以从以太坊网络获得,本文通过Etherscan提供的API来获得每一笔交易的执行结果。然而,不同的智能合约在以太坊中产生的交易数量差异较大,有些热门合约可能产生了成千上万笔交易(如0xab478a87c798789d80f800868667e),有些合约可能仅有3~5笔交易(如0xab478a87c798789d80f800868667e),本文对此情况的处理方法如下。
\section{实验设计与结果分析}

\subsection{实验设计}
分别使用静态代码指标和语义特征,

实验的运行环境、模型参数和结果评估指标等设置均遵循上一章节。
\subsection{实验结果分析}
基于特征融合的智能合约漏洞检测和运行时崩溃检测研究

漏洞检测预崩溃预测的关联