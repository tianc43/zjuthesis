\chapter{智能合约运行时崩溃预测的尝试性探索}
\section{本章内容}
\label{sec:本章内容5}
当前,关于智能合约在运行时发生异常而崩溃的研究寥寥无几,因此本文在前文建立的数据集和漏洞检测模型的基础上,对智能合约的运行时崩溃预测进行了研究。
\section{研究动机}
\label{sec:研究动机}
根据Siwapol等人的研究结果,截止2020年,以太坊网络中由于执行交易失败的合约仅手续费已经消耗了超过2800万美元,消耗的区块奖励更是达到1210亿美元。针对这一现状,如果能在发送交易前,根据当前状态信息和即将调用的智能合约源代码对交易结果进行预测,提前发现可能失败的交易,便能节省交易手续费,同时在一定程度提高以太坊的运行效率。

智能合约在运行时可能因为发生异常而导致交易失败,以太坊官方文档中表明了失败交易的执行状态包括:Gas不足、Revereted、无效操作码、无效跳转和堆栈溢出。当然这只是对于交易失败状态的简单表述,具体的合约发生异常的原因多种多样,但是值得注意的是,智能合约被部署到以太坊网络上后,其字节码便不可再更改,也即合约每次被调用后的执行逻辑是完全相同的。因此本文认为,除了合约被调用时的输入和一些简单的状态信息外,合约执行失败的原因与其源代码本身有很大的关系,而上文的智能合约漏洞检测工作中,实验结果证明了静态代码指标和语义特征能充分表达智能合约源代码的结构特征和语义信息。鉴于此,本文尝试利用静态代码指标和语义特征对智能合约运行时是否会导致交易失败进行探索。
\begin{table}[htbp]
    \caption{\label{tab:new_metrics}现有的智能合约漏洞检测方法}
    \small
            \renewcommand{\arraystretch}{1.5}
        \begin{tabularx}{\linewidth}{cp{3.5cm}X<{\raggedright}}
            \hline
            \textbf{维度}            & \textbf{指标}              & \textbf{描述} \\ \hline
            \multirow{6}{*}{代码复杂度指标(6个)} & AvgComplexity        & 所有函数的圈复杂度的平均值 \\
                                       & MaxComplexity        & 所有函数的圈复杂度的最大值 \\
                                       & SumComplexity        & 所有函数的圈复杂度的总和 \\
                                       & MaxInheritance   & 主合约继承树的最大深度 \\
                                       & MaxNesting           & 主合约中控制结构的最深级别 \\
                                       & CountContractCoupled  & 与主合约存在耦合关系的合约的数量 \\ \hline
            \multirow{5}{*}{面向对象维度(5个)} & CountContractBase    &  主合约直接继承的合约的数量 \\
                                       & CountDependence      & 主合约直接或间接依赖的合约总数 \\
                                       & CountTotalFunction   & 主合约中定义的函数的总数 \\
                                       & CountFunctionPublic  & 主合约中定义的公开函数的个数 \\
                                       & CountVariable        & 主合约中定义的变量的个数 \\ \hline
            \multirow{14}{*}{Solidity维度(14个)} & NOI                 & 主合约中使用的接口总数 \\
                                       & NOL                & 主合约中使用的库合约总数\\
                                       & NOSV                & 主合约中定义的存储变量总数 \\
                                       & NOMap              & 主合约中使用的映射类型总数 \\
                                       & NOPay               & 主合约中调用的可支付函数总数 \\
                                       & NOE                 & 主合约中定义的事件总数 \\
                                       & NOMod               &  主合约中定义的函数修改器总数 \\
                                       & NOT                 & 主合约的所有函数中执行的转账语句总数 \\
                                       & NOC                 & 主合约中消息调用总数 \\
                                       & NODC                & 主合约的所有函数中执行的委托调用总数 \\
                                       & SDFB                & 主合约是否重写了 fallback 函数 \\ 
                                       & NOReq         & 主合约中require语句的个数 \\
                                       & NOAst         & 主合约中assert语句的个数 \\
                                       & NORvt         & 主合约中revert语句的个数 \\ \hline
        \end{tabularx}
\end{table}

\section{实验数据集}
\label{sec:实验数据集}


% \textcolor{red}{这里可能要补充一些运行时的信息作为特征,不然可能会被喷:为什么同样的特征既可以用来检测漏洞也可以用来检测运行时异常?}

在数据集方面,尽管xxx节提出的34个静态代码指标能有效的表达程序的结构特征,但是对于智能合约运行时崩溃预测工作来说,有些指标对于合约执行逻辑的表达能力有限,因此本节特对34个静态代码指标中Solidity维度的指标进行补充,并且删去一些相关性不大的指标,最终得到了25个静态代码指标,具体如\autoref{tab:new_metrics}所示。

接下来要根据智能合约在以太坊中的交易情况对数据集进行标注,即确定一个合约是否会在被调用时发生异常而导致交易失败,本文称这些合约为崩溃合约。不同智能合约产生的交易数量差异较大,有些热门合约可能产生了成千上万笔交易\footnote{如地址为0x429881672b9ae42b8eba0e26cd9c73711b891ca5的智能合约},有些合约可能仅有3$\sim$5笔交易\footnote{如地址为0x3340e2f658a9820f8b23b773817d094cd2ece9ee的智能合约}。Chen等人进行了一项研究\cite{chen2020understand}发现大约81\%的账户(96\%的智能合约账户和77\%的外部账户)在以太坊上的交易少于5次,也就是说大多数账户(尤其是智能合约账户)并不频繁发起交易。此外,我们在BigQuery平台上对以太坊上智能合约及其交易信息进行了分析,结果表明94.3\%的合约的交易数在30以下(包含30),具体结果如表xxx所示。因此,我们选择30次交易作为基准对原数据集进行过滤,删去交易次数大于30的智能合约,最终我们得到了一个包含xxx份智能合约的数据集。下一步对数据集进行标注,如果一个智能合约的所有交易中只要产生了一次失败的交易,那么我们就标记这个合约为崩溃合约。以太坊中记录了智能合约的所有交易的执行结果,因此智能合约在运行时是否发生异常等信息都可以从以太坊网络获得,本文通过Etherscan提供的API来收集每一笔交易的执行结果,然后完成数据集的标注工作。最终,在xxx份合约中,有xxx份被标记为崩溃合约,接下来我们将使用此数据集开展智能合约运行时崩溃预测的实验。


% 对数据集进行统计性分析
% 在数据集方面,我们沿用了从前文搜集的合约中提取出的专家特征和语义特征,但在标签上有一些不同。

% 智能合约在运行时可能发生的异常如xxx节所述,包括Gas不足、reverted、无效操作码、无效跳转和堆栈溢出。
% 下文将在运行时刻发生异常而崩溃的合约称为崩溃合约。

% 以太坊中记录了智能合约的所有交易的执行结果,因此智能合约在运行时是否发生异常等信息都可以从以太坊网络获得,本文通过Etherscan提供的API来收集每一笔交易的执行结果。

% 然而,不同的智能合约在以太坊中产生的交易数量差异较大,有些热门合约可能产生了成千上万笔交易(如0xab478a87c798789d80f800868667e),有些合约可能仅有3~5笔交易(如0xab478a87c798789d80f800868667e),本文对此情况的处理方法如下。

% 这里对各种运行时异常进行分析
\section{实验设计与结果分析}
\label{sec:实验设计与结果分析2}

\subsection{实验设计}
\label{sec:实验设计}
实验设计与上一章节的内容大致相同,即先使用编写好的程序为每一个智能合约计算新的静态代码指标,然后对其生成数据流图、控制流图和函数调用图,再利用GraphCodeBERT模型对源代码和语义图生成高维度的语义特征,最后融合基于静态代码指标的专家特征和语义特征,并使用线性分类器进行类别预测。每一步的细节都可以参考上一章节中的内容。另外,实验的运行环境、模型参数和结果评估指标等设置均遵循上一章节中的设定。
\subsection{实验结果分析}
\label{sec:实验结果分析2}
基于特征融合的智能合约漏洞检测和运行时崩溃检测研究

漏洞检测预崩溃预测的关联

崩溃预测是最后一道防线,编写bug-free的程序才是最应该被重点关注的事,针对实验结果,提出以下几点针对性措施:

应对Gas不足的运行时异常,首先保证程序没有死循环

revert操作码,尽量减少会导致revert的操作

无效操作码和无效跳转,尽量减少

若 Gas 设置得过低,会导致交易失败,且你所支付的矿工费依然会被以太坊网络收取,不会退回至你的钱包。因此在设置 Gas 时,请设置得高一些以免交易失败损失矿工费。

较低的矿工费可能会使交易确认时间延长,由于部分去中心化应用对交易时间有限制,过长的等待时间可能导致交易失败。

使用经济挡可能导致交易上链延迟。交易被打包上链的时间主要由两个因素决定:矿工费高低和网络拥堵程度。当网络拥堵时,交易可能需要更长的时间才能被矿工打包进区块。如果设置的矿工费过低,矿工可能会优先选择矿工费更高的交易,导致交易等待时间过长。